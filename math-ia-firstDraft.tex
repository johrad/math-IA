\documentclass[12pt]{article}

\usepackage{amsmath}    % need for sub-equations
\usepackage{graphicx}   % need for figures
\usepackage{verbatim}   % useful for program listings
\usepackage{color}      % use if color is used in text
\usepackage{subfigure}  % use for side-by-side figures
\usepackage{hyperref}   % use for hypertext links, including those to external documents and URLs
% \usepackage[showframe]{geometry} % remove gay indentation 
% \usepackage{gensymb} % allows me to use fancy symbols in-text


\usepackage{graphicx} % use for images
\graphicspath{ {./images/} } % set path used for images

\setlength{\baselineskip}{16.0pt}    % 16 pt usual spacing between lines

\setlength{\parskip}{3pt plus 2pt}
\setlength{\parindent}{20pt}    
\setlength{\oddsidemargin}{0.5cm}
\setlength{\evensidemargin}{0.5cm}
\setlength{\marginparsep}{0.75cm}
\setlength{\marginparwidth}{2.5cm}
\setlength{\marginparpush}{1.0cm}
\setlength{\textwidth}{150mm}
\linespread{1.25}


 \begin{comment}
    \pagestyle{empty} % doesn't count for page numbers
 \end{comment}
        
        
\begin{document}
        
        \author{Candiate code: hjk123}            \begin{titlepage}
                \begin{center}
                  % \small{Mathematics SL, IA:} \\
                \Huge{A Mathematical Exploration of the \\ Decay of Caffeine}
                \break
                {\large A Mathematics Internal Assessment}
                \break                    %\small{Written by Johan-Petter R. Dragic}
                 % \date{06.11.2018}
                    
        
                    % ADD BREAK HERE
                \vspace{12mm}
                \includegraphics[scale=0.38]{coffee1.jpg}
                % source pic:http://www.mobileswall.com/wallpaper/coffee-beans-in-a-coffee-cup-free/            
    
            \end{center}
                    \end{titlepage}
        
            \tableofcontents
            \thispagestyle{empty}
            \addtocounter{page}{-1}
            
        %     Structure:
        %             1. Introduction
        %                 purpose and aim, personal engagement
        %             2. Collection of data
        %                 Experimental methods
                                % Simulation



            %%%%%%%%%%%%%%%%%%%%%%%%%%%%%    START OF DOCUMENT    %%%%%%%%%%%%%%%%%%%%%%%%%%%%%%%%%%%%%%%% 
            \newpage
            \section{Introduction} % and personal engagement
            \paragraph{}
                Coffee has become an integral part of many lives, including mine. It's what wakes you up in the morning and sustains your energy-levels throughout the day. The leading chemical stimulant in coffee is called caffeine and it's classified as a "Psychoactive drug", meaning that it alters the chemical processes that goes on in your brain. The main effect of caffeine is to alter the brain's perception of when it's tired, and is the reason why I and many others admire this mild drug. And since the world average amount of sleep lies bellow the hourly optimum, it is no wonder why caffeine consumed to such extents as it is today.
                \\
                Due to the psychoactive effects of caffeine on tiredness many advice to not drink caffeine-containing beverages 3 hours before bed, but how did they come to this conclusion?
                


                \noindent The purpose of this math IA is therefore to explore the decay of caffeine with the aim of figuring out how early before bed you can dink your last cup of coffee without it affecting your sleep.



        \section{Data collection}
        \paragraph{}
                Much like radioactive elements in physics, caffeine follows the same pattern that alpha decay does, meaning that it has a half-life. A Half-life in the realm of nuclear physics is defined as time taken for the radioactive activity of an element to halve itself. So if we were to have a sample of the element Plutonium-238 has an initial activity of $1$ and a half life of 2 hours. Then if you then wait for 2 hours, the activity is now going to be {$\frac{1}{2}$}. I have therefore concluded to use computer simulations as my data collection, and since there are no available simulations specifically for caffeine I will collect data from a simple python program which will apply the principles of the half-life (Raw code is attached to the appendix). 

                \noindent To obtain the half-life of caffeine a search of scholarly articles have been conducted. A study conducted by the American Society of Health-System Pharmacists has found that the half-life of caffeine lies varies between 3 to 5 hours in adults. This is subject to errors made by the original writer but for the sake of this exploration we will assume that his findings hold true. 
                
                \noindent
                Since the aim of this exploration is to determine how early before bed you can drink your last cup of coffee, we will take assume that the person drinking is at the lower echelon of caffeine decomposition, and caffeine wil therefore have a half-life of \textit{5 hours}. 
                
                \noindent the table bellow shows the simulation output. A note to the data is that the simulation will not calculate for values down to the absolute 0, but rather down do a 0 value of 0.01 (15 data points). This is due to the nature of Mathematics when continuously dividing. (the last output of the simulation before it returned 0.0 was $4.95\times 10^-324$)

                % Inserting table
                \begin{center}
                        \begin{tabular}{cc}
                        Time (h) & mg of caffeine \\
                        0 & 100.0 \\ 
                        1 & 50.0 \\ 
                        2 & 25.0 \\
                        3 & 12.5 \\ 
                        4 & 6.25 \\ 
                        5 & 3.125 \\ 
                        6 & 1.5625 \\
                        7 & 0.78125 \\
                        8 & 0.390625 \\
                        9 & 0.1953125 \\ 
                        10 & 0.09765625 \\ 
                        11 & 0.048828125 \\ 
                        12 & 0.0244140625 \\ 
                        13 & 0.01220703125 \\
                        14 & 0.006103515625 \\
                        
                                           
                        \end{tabular}
                        
                \end{center}
                

        
\end{document}

